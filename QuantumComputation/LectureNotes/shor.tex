% Created 2014-02-04 Tue 12:55
\documentclass[11pt]{article}
\usepackage[utf8]{inputenc}
\usepackage[T1]{fontenc}
\usepackage{fixltx2e}
\usepackage{graphicx}
\usepackage{longtable}
\usepackage{float}
\usepackage{wrapfig}
\usepackage{rotating}
\usepackage[normalem]{ulem}
\usepackage{amsmath}
\usepackage{textcomp}
\usepackage{marvosym}
\usepackage{wasysym}
\usepackage{amssymb}
\usepackage{hyperref}
\tolerance=1000
\author{Andrew Tulloch}
\date{\today}
\title{shor}
\hypersetup{
  pdfkeywords={},
  pdfsubject={},
  pdfcreator={Emacs 24.3.1 (Org mode 8.2.5f)}}
\begin{document}

\maketitle
\tableofcontents

\section{Step 1}
\label{sec-1}

Choose $1 < a < N$ at random.

Use the Euclidian algorithm to compute hfc(a, N).  If $a \neq 1$,
finished!


If $a = 1$, then \$a, N are coprime.

Euler's theorem gives that if a, N are coprime, then there is a
least power $1 < r < N$ such that $a^r \equiv 1 \mod N$.  $r$ is
called the "order of $a \mod N$".  So $f: \Z \rightarrow \Z_n$ with
$f(k) = a^k \mod N$ has period $r$.  $f$ is $1-1$ in each period\$.

$f$ is efficiently computable, since by repeated squaring we can
find \$f(k) in time poly log k.

It is classically hard to find $r$ in poly n time.

Suppose we can find \$r, and suppose $r$ is even.  Then by Legendre, we
have the result that

(a$^{\text{r}}$ - 1) = (a$^{\text{r}}$/2 + 1) (a$^{\text{r}}$/2 - 1) $\equiv$ 0 mod N

We can then find factors of \$N with gcd(N, a$^{\text{r}}$/2 \textpm{} 1).

Theorem if \$N is odd and not a prime power, and $a < N$ is chosend
uniformly at random with gcd(a, N) = 1\$, then $Prob(r even, a^r/2
  \neq -1 \mod N) \geq \frac{1}{2}$

So repeating the a choices say k times, the probability we fail to
fnid a given factor is bounded above by 1/2$^{\text{k}}$.

\section{How to find $r$}
\label{sec-2}
Use a period finding algorithm.  $f(k) = a^k \mod N$ has period \$r
on $Z$, for $r < N$.  Work on a finite domain \$D = \{0, 1, 2, \dots{},
2$^{\text{m}}$-1\} = Z$_{\text{2}^{\text{m}}}$.  For 2$^{\text{m}}$ being the least power of 2 greater than
$N^2$.

We have $2^m = Br = b$, 1 < b < r, so \$B > N full periods and one
corrupt period of length $b$.

Now apply a qunatum period finding algorithm, makin.

Make |f> = 1/sqrt\{2$^{\text{m}}$\} $\sum$ |x>|f(x)>, measure value.

Obtain |per> = 1/sqrt\{A\} $\sum$$_{\text{k}}$=1$^{\text{A-1}}$ |x$_{\text{0}}$ + kr>, apply the QFT to
obtain

1/sqrt\{2$^{\text{m}}$\} $\sum$$_{\text{c}}$=0$^{\text{2}}$$^{\text{m}}$-1 e$^{\text{2}\pi\ \text{i / 2}^{\text{m}}}$(x$_{\text{0}}$ + kr c) | c >

when $A = B$ or \$B + 1 depending on random x$_{\text{0}}$ vs $b$ value.

Then QFT$_{\text{2}}^{\text{m}}$ |per> = $\sum$$_{\text{c=0}}^{\text{2}^{\text{m}} \ \text{- 1}}$ \textasciitilde{} f(c) | c >

\section{}
\label{sec-3}
% Emacs 24.3.1 (Org mode 8.2.5f)
\end{document}