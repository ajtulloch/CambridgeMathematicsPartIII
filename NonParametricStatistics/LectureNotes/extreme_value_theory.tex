
\chapter{Extreme Value Theory}
\label{cha:extreme-value-theory}

Let $X_{n}$ be an IID sample from a distribution function $F$, and
denote $X_{(n)} = \max \{ X_{1}, \dots, X_{n} \}$ as the maximum order
statistic.

Without any normalization, $X_{(n)} \rightarrow x_{\star} = \inf \{ x:
F(x) = 1 \}$.

This is not overly interesting, since the limit distribution is
degenerate (we call $F$ non-degenerate if there does not exists $a \in
\R$ such that $F(x) = \I{x \geq a}$)

We may ask if there exists $\{ a_{n} \} > 0$, $\{ b_{n} \} > 0$, and a
nondegenerate $G$ such that
\begin{equation}
  \label{eq:118}
  \Prob{\frac{X_{(n)} - b_{n}}{a_{n}} \leq x} \rightarrow G(x)  
\end{equation} for all continuity points $x$ of $G$

Classical extreme value theory starts by asking:
\begin{enumerate}
\item What kind of $G$ appears in the limit of \eqref{eq:118}?
\item Cna we characterize $F$ such that \eqref{eq:118} holds for a
  specific limit distribution $G$?
\end{enumerate}

For the first question, we have the Extremal Types theorem. For the
second question, we have the ``domain of attraction'' problem.


\section{Prelimiaries}
\label{sec:prelimiaries}

Recall that $\Prob{X_{(n)} \leq x} = F(x)^{n}$.  We say that $F$ is in
the domain of attraction of $G$ ($F \in D(G)$) if there exists $\{ a_{n} \} > 0$, $\{
b_{n} \} $ and a non-degenerate $G$ such that
\begin{equation}
  \label{eq:119}
  \Prob{\frac{X_{(n)} - b_{n}}{a_{n}} \leq x} = [\text{$F(a_{n} x +
    b_{n})^{n} \rightarrow G(x)$ for all continuity points $x$ of $G$}].
\end{equation} and write $F(a_{n} x + b_{n})^{n} \hookrightarrow G(x)$.

We say that $G_{1}$ and $G_{2}$ are of same type if $G_{1}(ax + b) =
G_{2}(x)$ for some $a > 0, b$.

The next lemma shows that if $F \in D(G_{1})$ and $F \in D(G_{2})$,
then $G_{1}$ and $G_{2}$ are of the same type.

\begin{lem}
  Suppose $X_{n}$ is an IID sample from $F$ and there exists $\{ a_{n}
  \} > 0, \{ b_{n} \}$ and non-degenerate $G$ such that $F(a_{n} x
  +b_{n})^{n} \hookrightarrow G(x)$ .  Then there exists $\{ \alpha_{n} \} > 0,
, \{ \beta_{n} \}$ and non-degenerate $G_{\star}$ such that
$F(\alpha_{n} x _{ \beta_{n}})^{n} \hookrightarrow G_{\star}(x)$. if and
only if $\frac{\alpha_{n}}{a_{n}} \rightarrow a$ for some $a > 0$, and
$\frac{\beta_{n} - \beta}{a_{n}} \rightarrow b$ for some $b$.

Then we can let $G_{\star}(x) = G(ax + b)$.
\end{lem}

\begin{proof}
  See Galambos (1978), Lemma 2.2.3
\end{proof}

\begin{defn}
  \label{defn:extreme_value_theory:1}
  $G$ is \textbf{max-stable} if for every $n \in \N$, there exists
  $\{ a_{n} \} > 0, \{ b_{n} \} $ such that $G^{n}(a_{n}x + b_{n}) = G(x)$
\end{defn}

\begin{thm}
  \label{defn:extreme_value_theory:2}
  $D(g)$ is non-empty if and only if $G$ is max-stable.
\end{thm}

\begin{proof}
  (<=) If $G$ is max-stable, $G^{n}(a_{n} x + b_{n}) \hookrightarrow
  G(x)$.  Thus, by definitino, $G \in D(G)$.

  (=>) Let $F \in D(G)$. Then, there exists $\{ a_{n} \} > 0, \{ b_{n}
  \}$ such that $F^{n}(a_{n} x + b_{n}) \hookrightarrow G(x)$.  For
  each $k \in \N$, we replace $n$ by $nk$, and then
  \begin{align}
    \label{eq:121}
    F^{nk}(a_{nk} x + b_{nk}) \hookrightarrow G(x) 
  \end{align}
  Thus $F^{n}(a_{nk} x + b_{nk}) \hookrightarrow G^{\frac{1}{k}}(x)$.
  Since $G^{\frac{1}{k}}$ is also non-degenerate, $G^{\frac{1}{k}}(x)
  = G(a_{k} x + b_{k})$, which implies $G(x) = G^{k}(a_{k} x + b_{k})$
  as they are of the same type.
\end{proof}

\begin{thm}
  \label{defn:extreme_value_theory:3}
  If $F \in D(G)$, then $G$ must belong to the following distributions
  (within type):

  \begin{enumerate}
  \item Frechet - $G_{1, \alpha}(x) = \exp(-x^{-\alpha})$, $x > 0$,
    $\alpha > 0$
  \item Negative Weibull - $G_{2, \alpha} = \exp(- (-x)^{\alpha})$, $x
    < 0$, $\alpha > 0$
  \item Gumbel - $G_{3}(x) = \exp(-\exp(-x))$, $x \in \R$.
  \end{enumerate}

  Conversely, these distributions can appear as such limits in \eqref{eq:118}.
\end{thm}

\begin{remark}
  \begin{enumerate}
  \item Using $X_{(1)} = -\max \{ -X_{1}, \dots, -x_{n} \} $, we have
    equivalent theorems in terms of normalized minima.
  \item Sometimes, we cannot have nondegenreate $G$ of normalized
    maxima  - for example $X_{1}, \dots, X_{n} \sim
    Bern(\frac{1}{2})$, $X_{(n)}$.
  \item We can combine these three types into Generalized Extreme
    Value Distribution (GEV) -
    \begin{equation}
      \label{eq:122}
      G(x; \mu, \sigma, \gamma) = \exp(- (1 + \gamma(\frac{x-\mu}{\sigma}))^{-\frac{1}{\gamma}})
    \end{equation} with $1 + \gamma(\frac{x - \mu}{\sigma}) > 0$, $\mu
    \in \R, \gamma \in \R, \sigma > 0$.
  \end{enumerate}

  We have Frechet corresponds to $\gamma > 0$, $\alpha =
  \frac{1}{\gamma} $, NW is $\gamma < 0$, $\alpha =
  -\frac{1}{\gamma}$, and gumbel correpsonds to the case where $\gamma
  \rightarrow 0$.
\end{remark}

\begin{proof}[Proof (non-examinable)]
  We show $Y_{n} = \frac{X_{(n)} - b_{n}}{a_{n}} \cd Y$, with
  $G_{\gamma}(x) = \exp(- (1 + rx)^{-\frac{1}{r}})$

  Then, using Helly's theorem, we have $\E{z(Y_{n})} \rightarrow
  \E{z(Y)}$ for all continuous bounded $z$.  Then the LHS is given by 
  \begin{align}
    \label{eq:123}
    \int_{}^{} z \frac{x - b_{n}}{a_{n}} d F_{X_{(n)}}(x) = n
    \int_{}^{} z(\frac{x - b_{n}}{a_{n}}) F(x)^{n-1} dF(x)
  \end{align} and changinge variables so $F(x) = 1 - \frac{v}{n}, x = ..$
  
\end{proof}


%%% Local Variables: 
%%% mode: latex
%%% TeX-master: "master"
%%% End: 
