\section{ARIMA Processes}
\label{sec:arima-processes}

\begin{defn}
  \label{sec:arima-processes-1}
  If $d$ is a non-negative integer, $X_{t}$ is said to be an
  ARIMA($p,d,q$) process if $Y_{t} = (1 - B)^{d} X_{t}$ is a causal
  ARMA($p, q$) process
\end{defn}

\begin{thm}
  \label{sec:arima-processes-2}
  ARIMA models should be used when there is a slowly decaying positive
  sample autocorrelation function.

  If there is a slowly decaying oscillatory  sample ACF, applying
  the operator $(1 - B + B^{2})$ can applied to produce a series with
  a more rapidly autocorrelation function.
\end{thm}

\subsection{The Box-Cox Transformation}
\label{sec:box-cox-transf}

\begin{defn}
  \label{sec:arima-processes-3}
  The Box-Cox transformation is a\textbf{variance-stabilizing
    transformation} - and should be used whenever the standard
  deviation increases \textbf{linearly} with the mean. The equation is
  \begin{equation}
    \label{eq:77}
    f_{\lambda}(U_{t}) =
    \begin{cases}
      \frac{(U_{t} - 1)}{\lambda}  & U_{t} \geq 0, \lambda > 0 \\
      \ln U_{t} & U_{t} > 0, \lambda = 0.
    \end{cases}
  \end{equation}
\end{defn}

\subsection{Unit Roots Test}
\label{sec:unit-roots-test}

\begin{thm}
  \label{sec:unit-roots-test}
  Let $X_{1}, \dots, X_{n}$ be observations from the AR($1$) model
  $X_{t} - \mu = \phi_{1}(X_{t -1} - \mu) + Z_{t}$ for $Z_{t} \sim
  WN(0, \sigma^{2})$ where $|\phi_{1} < 1$ and $\E{X_{t}} = \mu$.  For
  large $n$, the maximum likelihood estimator of $\hat \phi_{1}$ is
  approximately $N(\phi_{1}, \frac{1 - \phi_{1}^{2}}{n} )$.  In the
  unit root case, the normal approximation is no longer applicable,
  which precludes its use for testing the unit root hypothesis $H_{0}:
  \phi_{1} = 1$ vs $H_{1}: \phi_{1} < 1$.

  To construct a test of $H_{0}$, write the model as $\deltaX_{t} =
  X_{t} - X_{t-1} = \phi^{\star}_{0} + \phi^{\1} X_{t-1} + Z_{t}$,
  $Z_{t} \sim WN(0, \sigma^{2})$ where $\phi^{\star}_{0} = \mu(1 -
  \phi_{1})$ and $\phi^{\star}_{1} = \phi_{1} - 1$.  Now, let $\hat
  \phi_{1}^{\star}$ be the OLS estimator of $\phi_{1}^{\star}$ found
  by regression $\delta X_{t}$ on $1$ and $X_{t-1}$.  The estimated
  standard error of $\phi_{1}^{\star}$ is
  \begin{equation}
    \label{eq:77}
    \hat{SE} (\hat \phi_{1}^{\star}) = S(\sum_{t=1}^{n} (X_{t-1} -
    \overline X)^{2})^{-\frac{1}{2}}
  \end{equation} where $S^{2} = \sum_{t=2}^{n} (\delta X_{t} - \hat
  \phi_{0}^{\star} - \hat \phi_{1}^{\star} X_{t-1})^{2} / (n - 3)$ and
  $\overline X$ is the sample mean of $X_{1}, \dots, X_{n-1}$.

  Then the limit distribution of the $t$-ration $\hat \tau_{\mu}=
  \frac{\hat \phi_{1}^{\star}}{\hat{SE}(\hat \phi_{1}^{\star})}$ under
  the unit root assumption can be derived.
\end{thm}

\begin{thm}
  \label{sec:unit-roots-test}
  Let $X_{t}$ be a causal and invertible ARMA($p, q$) process
  satisfying $\phi(B)X_{t} = \theta(B) Z_{t}$, $Z_{t} \sim WN(0,
  \sigma^{2})$.  Then the differenced series $Y_{t} = \Delta X_{t}$ is
  a non-invertible ARMA($p, q+1$) process with moving average
  polynomial $\theta(z)(1-z)$.  Thus, testing for a unit root in a MA
  polynomial is equivalent to testing that the time series has not been
  over-differenced.

  Let $X_{1}, \dots, X_{n}$ be observations from the MA(1) model $X_{t}
  = Z_{t} + \theta Z_{t-1}$, $Z_{t} \IID(0, \sigma^{2})$.  Then under
  the assumption $\theta = -1$, $n(\hat \theta + 1)$ where $\hat
  \theta$ is the MLE converges in distribution.  A test of $H_{0}:
  \theta = -1$ vs $H_{1}: \theta > -1$ can be fashioned on this
  limiting result by rejecting $H_{0}$ when $\hat \theta > -1 +
  \frac{c_{\alpha}}{n}$ where $c_{\alpha}$ is the $(1 - \alpha)$
  quantile of the limit distribution of $n(\hat \theta + 1)$.
\end{thm}


%%% Local Variables: 
%%% mode: latex
%%% TeX-master: "master"
%%% End: 
