\section{Spectral Analysis}
\label{sec:spectral-analysis}

Let $X_{t}$ be a zero-mean stationary time series with autocovariance
function $\gamma(\cdot)$ satisfying $\sum_{h=-\infty}^{\infty}
|\gamma(h)| < \infty$.

\begin{defn}
  \label{defn:spectral_analysis:1}
  The spectral density of $X_{t}$ is the function $f(\cdot)$ defined
  by
  \begin{align}
    \label{eq:47}
    f(\lambda) = \frac{1}{2 \pi} \sum_{h=-\infty}^{\infty}
    e^{-ih\lambda} y(h)
  \end{align}  The summability implies that the series converges
  absolutely.
\end{defn}

\begin{thm}
  \label{defn:spectral_analysis:2}
  \begin{enumerate}
  \item $f$ is even
  \item $f(\lambda) \geq 0$ for all $\lambda \in (-\pi, \pi]$.
  \item $\gamma(k) = \int_{-\pi}^{\pi} e^{-k\lambda} f(\lambda) d
    \lambda = \int_{-\pi}^{\pi} \cos (k \lambda) f(\lambda) d\lambda$.
  \end{enumerate}
\end{thm}

\begin{defn}
  \label{defn:spectral_analysis:3}
  A function $f$ is the \textbf{spectral density} of a stationary time
  series $X_{t}$ with autocovariance function $\gamma(\cdot)$ if
  \begin{enumerate}
  \item $f(\lambda) \geq 0$ for all $\lambda \in (0, \pi]$,
  \item $\gamma(h) = \int_{-\pi}^{\pi} e^{ih\lambda} f(\lambda)
    d\lambda$ for all integers $h$.
  \end{enumerate}
\end{defn}

\begin{lem}
  If $f$ and $g$ are two spectral density corresponding to the
  autocovariance function $\gamma$, then $f$ and $g$ have the same
  Fourier coefficients and hence are equal.
\end{lem}

\begin{thm}
  \label{defn:spectral_analysis:4}
  A real-valued function $f$ defined on $(-\pi, \pi]$ is the spectral
  density of a stationary process if and only if
  \begin{enumerate}
  \item $f(\lambda) = f(-\lambda)$,
  \item $f(\lambda) \geq 0$
  \item $\int_{-\pi}^{\pi} f(\lambda) d\lambda < \infty$.
  \end{enumerate}
\end{thm}

\begin{thm}
  \label{defn:spectral_analysis:5}
  An absolutely summable function $\gamma(\cdot)$ is the
  autocovariance function of a stationary time series if and only if
  it is even and
  \begin{equation}
    \label{eq:70}
    f(\lambda) = \frac{1}{2 \pi} \sum_{h=-\infty}^{\infty}
    e^{-ih\lambda} \gamma(h) \geq 0
  \end{equation} for all $\lambda \in (-\pi, \pi]$, in which case
  $f(\cdot)$ is the spectral density of $\gamma(\cdot)$.
\end{thm}

\begin{thm}[Spectral Representation of the ACVF]
  \label{defn:spectral_analysis:6}
  A function $\gamma(\cdot)$ defined on the integers is the ACVF of a
  stationary time series if and only if there exists a
  right-continuous, nondecreasing, bounded function $F$ on $[-\pi,
  \pi]$ with $F(-\pi) = 0$ such that
  \begin{equation}
    \label{eq:72}
    \gamma(h) = \int_{-\pi}^{\pi} e^{ih\lambda} dF(\lambda)
  \end{equation} for all integers $h$.
\end{thm}

\begin{remark}
  The function $F$ is a \textbf{genrealized distribution function} on
  $[-\pi, \pi]$ in the sense that $G(\lambda) =
  \frac{F(\lambda)}{F(\pi)}$ is a probability distribution function on
  $[-\pi, \pi]$. Note that since $F(\pi) = \gamma(0) = \Var{X_{1}}$,
  the ACF of $X_{t}$ has the spectral representation function
  \begin{equation}
    \label{eq:73}
    \rho(h) = \int_{-\pi}^{\pi} e^{ih \lambda} dG(\lambda)
  \end{equation}

  The function $F$ is called the spectral distribution function of
  $\gamma(\cdot)$. If $F(\lambda)$ can be expressed as $F(\lambda) =
  \int_{-\pi}^{\lambda} f(y) dy$ for all $\lambda \in [-\pi, \pi]$,
  then $f$ is the spectral density function and the time series is
  said to have a continuous spectrum. If $F$ is a discrete function,
  then the time series is said to have a discrete spectrum.
\end{remark}

%%% Local Variables: 
%%% mode: latex
%%% TeX-master: "master"
%%% End: 
