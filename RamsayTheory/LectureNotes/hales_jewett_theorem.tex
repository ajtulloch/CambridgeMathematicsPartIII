\chapter{The Hales-Jewett Theorem}
\label{cha:hales-jewett-theorem}

Let $X$ be a finite set.  Subset of $X^{n}$ is a \textbf{line} or
\textbf{combinatorial line} if there exists $I \subset [n]$, $I \neq
\emptyset$, and $a_{i} \in X$ for each $i \in [n] - I$, such that
\begin{equation}
  \label{eq:3}
  L = \{ x \in X^{n} | x_{i} = a_{i} \forall i \notin I, x_{j} = x_{k}
  forall j, k \in I \}
\end{equation}

\begin{thm}
  \label{defn:hales_jewett_theorem:1}
  Let $m, k \in \N$.  Then there exists $n \in \N$ such that whenever
  $[m]^{n}$ is $k$-coloured there exists a monochromatic line.
\end{thm}

\begin{remark}
  \begin{enumerate}
  \item The smallest such $n$ is denoted $HJ(m, k)$.
  \item So $m$-in-a-row naughts and crosses played in enough
    dimensions cannot end in a draw.\sidenote{Exercise: show that
      first-player winers.}
  \item Hales-Jewett implies Van Der Waerden's theorem.
    Indeed, given a $k$-colouring on $\N$, induce a $k$-colouring of
    $[m]^{n}$ by $c'((x_{1}, x_{2}, \dots, x_{n}) = c(x_{1} + x_{2} +
    \dots + x_{n})$. By Hales-Jewett, there exists a monochromatic line
    for $c'$.
  \end{enumerate}
\end{remark}



%%% Local Variables: 
%%% mode: latex
%%% TeX-master: "master"
%%% End: 
