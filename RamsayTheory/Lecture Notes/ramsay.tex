\chapter{Monochromatic Systems}
\label{cha:monochr-syst}

\begin{thm}[Ramsay's theorem]
  \label{defn:ramsay:3}
  Whenever $\N^{(2)}$ is two-coloured, there exists an infinite
  monochromatic set.
\end{thm}

\begin{enumerate}
\item Called a ``two-pass'' proof.
\item Same proof that whenever $N^{(2)}$ is $k$-coloured.
  Alternatively, view color as 1 and ``2 or 3 or ... or $k$''.   and
  by theorem one we get an infinite set of colour 1 - then just induct
  on $k$.
\item Having an infinite monochromatic set is stronger than asking for
  an arbitrarily large finite monochromatic set.
\end{enumerate}

\begin{exmp}
  \label{defn:ramsay:1}
  Any sequence $x_{1}, x_{2}, \dots$ in $\R$ (or any totally ordered
  set) has a monotone subsequence.
\end{exmp}

\begin{proof}
  Color \textbf{up} if $x_{i} < x_{j}$, \textbf{down} if $x_{i} \geq
  x_{j}$, and apply Theorem \ref{defn:ramsay:3}. 
\end{proof}

What about $\N^{(r)}, r = 3, 4, \dots$.  If we two-color $\N^{(r)}$,
can we get an infinite monochromatic set?

For example, consider $n=3$.  Color $N^{(3)}$ by colouring $(i,j,k)$
\textbf{red} if $i$ divides $j + k$, \textbf{blue} if not.

\begin{thm}[Ramsey's theorem for $r$-sets]
  \label{defn:ramsay:2}
  Whenever $\N^{(r)}$ is two-coloured, there exists an infinite
  monochromatic set.
\end{thm}

\begin{proof}
  Induction on $r$.  $r = 1$ is trivial by the pigeonhole principle.
  $r = 2$ is shown by Theorem \ref{defn:ramsay:3}.

  Now, given a two-colouring of $N^{(r)}$.  Choose $a_{1} \in \N$.  We
  induce a two-colouring $c'$ of $(\N - \{ a_{1} )^{(r-1)}$ by $c'(F)
  = c(F \cup \{ a_{1} \})$  for all $F \in (\N - \{ a_{1} \})^{(r -
    1)}$.  By induction, there exists an infinite monochromatic set
  $B_{1} \subseteq N - \{ a_{1} \}$ for $c'$.

  So all $r$-sets $F \cup \{ a_{1} \}$, $F \subset B_{1}$ have the
  same color ($c_{1}$, say).  Choose $a_{2} \in B_{1}$.  By the same
  argument, there exists an infinite set $B_{2} \subset B_{1} - \{
  a_{2} \}$ such that all $r$-sets $F \cup \{ a_{2} \}$, $F \subset
  B_{2}$ have the same colour.  Continue inductively.  We obtain a
  sequence of points $a_{1}, a_{2}, \dots$ and colors $c_{1}, c_{2},
  \dots$ such that each $r$-set $a_{i_{1}}, \dots, a_{i_{r}}$ with
  $i_{1} < \dots < i_{2}$ has color $c_{i_{1}}$.  But we must have
  $c_{i_{1}} = c_{i_{2}} = c_{i_{3}} = \dots$ for some infinite
  subsequence.  Then $\{ a_{i_{1}}, a_{i_{2}}, \dots \}$ is an
  infinite monochromatic sequence.
\end{proof}

\begin{exmp}
  \label{defn:ramsay:4}
  We can show that given any $(1, x_{1}), (2, x_{2}), \dots$ we can
  find a subsequence inducing a monotone function.  Consider the
 three-colouring of $(1, x_{1}), (2, x_{2}), (3, x_{3}),
  \dots$ by colouring triples of points \textbf{convex} or
  \textbf{convex} depending on the colouring of the set.
\end{exmp}

\begin{thm}
  \label{defn:ramsay:5}
  Infinite Ramsey (Theorem \ref{defn:ramsay:2}) implies the finite
  version.  That is, for all $m, r \in \N$, whenever $[m]^{(r)}$ is
  two-coloured there exists a monochromatic $m$-set.
\end{thm}

\begin{proof}
  Suppose not, so for all $n \geq r$ there exists a two-colouring
  $c_{n}$ of $[m]^{(r)}$ without a monochromatic $m$-set.  We'll
  construct a 2-colouring of $\N^{(r)}$ without a monochromatic
  $m$-set, contradicting Theorem \ref{defn:ramsay:2}.\sidenote{
    If the $c_{n}$ nested - that is, if $c_{n}|_{[n-1]^{(r)}} =
    c_{n-1}$, can take union, but they may \textbf{not} be nested
  }

  There are only finitely many ways to two-color $[r]^{(r)}$ (two, in
  fact). So infinitely many of the $c_{n}$ agree on $[r + 1]^{(r)}$.
  Say, $c_{i} | [r+1]^{(r)} = d_{r+1}$. Now, 
  \begin{enumerate}
  \item the $d_{i}$ are nested, and
  \item no $d_{n}$ has a monochromatic $m$-set (as there is some $k$
    such that $d_{n}$ = $c_{k} | [n]^{(r)}$.
  \end{enumerate}

  Define a colouring $c: \N^{(r)} \rightarrow [2]$ by $c(F) =
  d_{n}(F)$ for any $n \geq \max F$.  We obtain our contradiction.
\end{proof}

\begin{remark}
  \begin{enumerate}
  \item Proof gives no bound on what $n = n(m, r)$ we could take.
    There \textbf{are} direct proofs that do give upper bounds.
  \item Called a ``compactness argument''.  Essentially, we are
    proving that the space $[0, 1]^{\N}$ (all infinite 0-1 sequences)
    with the product topology (e.g. the metric $d(f, g) =
    \frac{1}{\min(n: f_{n} \neq g_{n})}$ is (sequentially) compact.
  \end{enumerate}
\end{remark}

What if we coloured $\N^{(2)}$ with $\infty$ many colours (i.e. w have
$c: \N^{(2)} \rightarrow X$ for some set $X$).  Obviously, we cannot
find an infinite $M$ on which $c$ is constant - for example, let $c$
be injective.

Can we always find an infinite $M$ such that $c$ is either constant on
$M^{(2)}$ or injective on $M^{(2)}$?  No - for example, $1 \mapsto
\{2,3,4, \dots \}, 2 \mapsto \{ 3, 4, 5, \dots \}, \dots$ as different
colours as a counterexample.

\begin{thm}[Canonical Ramsey Theorem]
  \label{defn:ramsay:6}
  Let $c: \N^{(2)} \rightarrow X$ for some set $X$.  Then there exists
  an infinite $M \in \N$ such that one of the following holds:
  \begin{enumerate}
  \item \label{item:1}$c$ is constant on $M^{(2)}$,
  \item \label{item:2}$c$ is injective on $M^{(2)}$,
  \item \label{item:3}$c(i, j) = c(k, l) \iff i = k$ with ($i, j, k, l \in M, i < j, k < l$)
  \item \label{item:4}$c(i, j) = c(k, l) \iff j = l)$ with ($i, j, k, l \in M, i < j, k < l$)
  \end{enumerate}
\end{thm}

\begin{remark}
  This generalizes enormously Theorem \ref{defn:ramsay:3} - if $X$ is
  finite then \ref{item:1}, \ref{item:3}, \ref{item:4} cannot arise.
\end{remark}

\begin{proof}
  We'll apply this for Ramsey's theorem on 4-sets.  Two-colour
  $\N^{(4)}$ by giving $(i,j,k,l)$ colour \textbf{same} if $c(i, j) =
  c(k, l)$, \textbf{different} otherwise.

  By Ramsey's theorem for 4-sets (Theorem \ref{defn:ramsay:2}), there
  exists an infinite set $M_{1}$ that is monochromatic for this
  colouring.

  If $M_{1}$ is coloured \textbf{same}, for any $i, j$ and $k, l$ in
  $M_{1}^{2}$, choose $m, n \in M_{1}^{(2)}$ within $m > j, l$.  Then
  $c(i, j) = c(m, n)$ and $c(k, l) = c(m, n)$.  So $c(i, j) = c(k, l)$
  so $c$ is constant on $M_{1}^{(2)}$.

  So now, we may assume $M_{1}$ is coloured differently.  Now
  two-colour $M_{1}^{(4)}$ by giving $(i,j,k,l)$ \textbf{same} if
  $c(i, l) = c(j, k)$, \textbf{different} otherwise. By Theorem
  \ref{defn:ramsay:2}, there exists an infinite set $M_{2} \subset
  M_{1}$ that is monochromatic for this colouring.

  If $M_{2}$ are coloured the same, choose $i < j < k < l < m < n$ in
  $M_{2}$. Then $c(j, k) = c(i, n)$ and $c(l, m) = c(i, n)$, whence
  $c(j, k) = c(l, m)$, which is a contradiction, as $M_{2} \subset
  M_{1}$.
  Thus, $M_{2}$ is coloured \textbf{different}.

  Two-colour $M_{2}^{(4)}$ by giving $(i,j,k,l)$ colour \textbf{same}
  if $c(i, k)  = c(j, l)$, \textbf{different} otherwise.  We have an
  infinite monochromatic colouring $M_{3} \subset M_{2}$ for this
  colouring.  If $M_{3}$ coloured \textbf{same}, choose $i < j < k < l
  < m < n$ in $M_{3}$.  Then $c(i,l) = c(j,m)$ and $c(i, l) = c(k,
  m)$, so $c(j, n) = c(k, m)$, a contradiction.  So $M_{3}$ is
  coloured \textbf{different}.

  Two-colour $M_{3}^{(3)}$ by giving $(i,j,k)$ colour \textbf{same} if
  $c(i,j) = c(j, k)$, \textbf{different} otherwise.  We have an
  infinite monochromatic sequence $M_{4} \subset M_{3}$ for this
  colouring.  If $M_{4}$ is coloured same, choose $i < j < k < l$ in
  $M_{4}$.  Then $c(i, j) = c(j, k) = c(k, l)$, a contradiction.  So,
  $M_{4}$ is coloured \textbf{differently}.

  Two-colour $M_{4}^{(3)}$, by giving $(i, j, k)$ colour
  \textbf{left-same} if $c(i, j) = c(i, k)$, \textbf{left-different}
  otherwise. We have an infinite monochromatic set $M_{5} \subset
  M_{4}$ for this.  Then two-colour $M_{5}^{(3)}$ by giving $(i, j,
  k)$ colour \textbf{right-same} if $c(j, k) = c(i, k)$,
  \textbf{right-different} if not.  We get an infinite monochromatic
  sequence $M_{6}$ for this colouring.

  If $M_{6}$ is \textbf{left-different, right-different}, we have case
  \ref{item:3}.  If $M_{6}$ is \textbf{left-same, right-different}, we
  have case \ref{item:2}.  If $M_{6}$ is \textbf{left-different,
    right-same}, we have case \ref{item:4}.  If $M_{6}$ is
  \textbf{left-same, right-same}, choosing $i < j < k$ in $M_{6}$,
  then $c(i, j) = c(i, k) = c(j, k)$, which is a contradiction.
\end{proof}

\begin{remark}
  \begin{enumerate}
  \item Could use just one colouring, according to the pattern of
    colours on the 2-sets inside a given 4-set.
  \item For any $r$, one can show similarly.  For \textbf{any}
    colouring $c$ of $\N^{(r)}$, there exists an infinite $M \subset
    \N$ and $I \subset [r]$ such that for all $i_{1} < \dots < i_{r}$
    and $j_{1} < \dots < j_{r}$ in $M$, $c(i_1, \dots, i_{r}) =
    c(j_{1}, \dots, j_{r}) \iff i_{n} = j_{n}$ for all $n \in I$.
    These $2^{r}$ colourings are the canonical colourings of
    $\N^{(r)}$.  

    For example, let $r = 2$.  $I = \{ 1 \}$ is case \ref{item:3}.
    $I = \{ 2 \}$ is case 
  \end{enumerate}
\end{remark}

%%% Local Variables: 
%%% mode: latex
%%% TeX-master: "master"
%%% End: 
