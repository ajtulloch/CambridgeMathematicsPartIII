\chapter{Example Sheet 1}
\label{cha:example-sheet-1}

\newcommand{\dual}{\overset{\star}\equiv}

\begin{enumerate}
\item \label{item:1}
  The Lagrangian of the problem is
  \begin{equation}
    \label{eq:4}
    L(x, \lambda) = -2x_{1}^{2} - x_{2}^{2} + x_{1} x_{2} -
    \lambda(3x_{1} + x_{2} - 10)
  \end{equation} with partial derivatives
  \begin{align}
    \label{eq:5}
    \frac{\partial L}{\partial x_{1}} = -4x_{1} + x_{2} + 8 - 3\lambda \\
    \label{eq:6}
    \frac{\partial L}{\partial x_{2}} = -2x_{2} + x_{1} + 3 - \lambda \\ \\
    \label{eq:7}
    \frac{\partial L}{\partial \lambda} = 3x_{1} + x_{2} - 10
  \end{align}

  and second partial derivatives
  \begin{align}
    \label{eq:8}
    \frac{\partial^{2} L}{\partial x_{1}^{2}} = -4 \\
    \frac{\partial^{2} L}{\partial x_{2}^{2}} = -2 \\
    \frac{\partial^{2} L}{\partial x_{1} \partial x_{2}} = 1 
  \end{align}

  Setting \eqref{eq:5}, \eqref{eq:6}, \eqref{eq:7} to zero and solving
  the set of equations gives
  \begin{align}
    \label{eq:9}
    x_{1} = \frac{69}{28} \\
    x_{2} = \frac{73}{28}
  \end{align}

  The Hessian matrix is
  \begin{equation}
    \label{eq:10}
    \begin{vmatrix}
      -4 & 1 \\
      1 & -2 \\
    \end{vmatrix} = 7 > 0
  \end{equation}
  and $\frac{\partial^{2} L}{\partial x_{1}^{2}} < 0$, so the solution
  is the maximizer of $L$.  Thus, by the Lagrangian Sufficiency
  theorem, the solution is a maximizer of original problem.
  
\item \label{item:2}
  
\item \label{item:3}
  
\item \label{item:4}
  Let $A \dual B$ indicate that $A$ is the dual of $B$.  Then we have
  \begin{equation}
    \label{eq:2}
    \min \{ c^{T}x | Ax \geq b, x \geq 0 \} \dual \max \{ b^{T} x |
    A^{T}x \leq c, x \geq 0 \}
  \end{equation}

  Let a linear program be given in the general form. Then
  \begin{align}
    \label{eq:3}
    \min \{ c^{T} x | Ax \geq b, x \geq 0 \} &\dual \max \{ b^{T} x |
    A^{T}x \leq c, x \geq 0 \} \\
    &= \min \{ (-b)^{T} x | (-A)^{T} x \geq (-c), x \geq 0 \} \\
    &\dual \{ (-c)^{T} x | ((-A)^{T})^{T} x \leq -b, x \geq 0 \} \\
    &= \min \{ c^{T}x | Ax \geq b, x \geq 0 \}
  \end{align}
  and thus the dual of the dual of a linear program is identical to
  the linear program.
\item \label{item:5}
  \begin{enumerate}
  \item The general linear program
    \begin{equation}
      \label{eq:11}
      \min \{ c^{T}x | Ax \geq b, x \geq 0 \} \dual \max \{ b^{T} x |
      A^{T}x \leq c, x \geq 0 \}
    \end{equation}

    We then have
    \begin{align}
      \label{eq:12}
      \max \{ 0^{T} x | Ax = b, x \geq 0 \}  &\equiv \min \{ (-0)^{T}
      x | Ax = b, x \geq 0 \} \\
      \label{eq:13}
      &\dual \max \{ z^{T}b | z^{T} A \leq 0^{T} \} \\
      \label{eq:14}
      &\equiv \min \{ y^{T}b | y^{T} A \geq 0^{T} \}
    \end{align}


    Thus, the dual of \eqref{eq:12} is \eqref{eq:14} as required.
  \item By weak duality, we have for any feasible solutions $y$ of
    \eqref{eq:12} and $x$ of \eqref{eq:14}, that
    \begin{equation}
      \label{eq:15}
      0^{T}x = 0 \leq y^{T}b
    \end{equation}

    If \eqref{eq:12} is feasible, then by weak duality, \eqref{eq:14}
    is bounded below by zero (by \eqref{eq:15}).

    \eqref{eq:14} is obviously feasible, since $y = 0$ is a feasible
    solution.  If \eqref{eq:14} is bounded, we can solve it using the
    Simplex method, which also gives us an optimal and therefore
    feasible solution for \eqref{eq:12}.
  \item The two cases correspond to the two cases of the above result.
    \eqref{eq:12} is feasible if and only if there exists $x \in
    \R^{n}$ with $Ax = b$ and $x \geq 0$. \eqref{eq:12} is infeasible
    if and only if \eqref{eq:14} is unbounded if and only if there
    exists $y \in \R^{m}$ such that $y^{T}A \geq 0$ and $y^{T}b < 0$.
  \end{enumerate}
\item \label{item:6}
  
  
\item \label{item:7}
  
\item \label{item:8}

\item \label{item:9}
  
\item \label{item:10}
\item \label{item:11}
  \begin{enumerate}
  \item
    Trivial to verify this problem is in \np, as given a certificate,
    simply follow the path through the nodes, which is linear in the
    number of nodes.
  \item
    Pretty tricky, just following construction in
    \url{HTTP://AMA.epfl.ch/~moustafa/Other/Complexityslides/lec7.pdf}.
  \item
    Traveling salesman problem - given an instance of undirected
    Hamiltonian cycle $G(V, E)$, construct an instance of TSP $T(V', d)$ by
    placing a city at each $v \in V$ (so $V' = V$), and let
    \begin{equation}
      \label{eq:1}
      d(v_{1}, v_{2}) =
      \begin{cases}
        1 & (v_{1}, v_{2}) \in E \\
        2 & \text{otherwise}
      \end{cases}
    \end{equation}

    There exists a TSP path of length $|V|$ if and only if there exists an
    undirected Hamiltonian cycle in the original graph.

    We now reduce directed Hamiltonian cycle to undirected Hamiltonian
    cycle. For an instance $G(V, E)$ of the directed problem,
    construct the undirected graph $G'(V', E')$ where each vertex $v_{i}$ is
    replaced by three vertices $v'_{i1}, v'_{i2}, v'_{i3}$, with edges
    $v'_{i1} \leftrightarrow v'_{i2}, v'_{i2} \leftrightarrow v'_{i3}$,
    and $v'_{i3} \leftrightarrow v'_{j1}$ if $v'_{i} \rightarrow
    v'_{j}$.

    We then trivially have that any undirected Hamiltonian cycle in
    $G'$ can be mapped to a directed Hamiltonian cycle in $G$, as
    required.
  \end{enumerate}
  
\end{enumerate}
%%% Local Variables: 
%%% TeX-master: "master""
%%% End: 
