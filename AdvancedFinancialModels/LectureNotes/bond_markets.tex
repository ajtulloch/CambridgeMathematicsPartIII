\chapter{Bond Markets and Interest Rates}
\label{cha:bond-mark-inter}

\begin{defn}
  \label{defn:bond_markets:1}
  A zero coupon bond is a contingent claim that pays exactly one unit
  of money at maturity.

  We assume that $\xi_{T}$, the payment of the bond, is 1 a.s. - that
  is, there is no credit risk.
\end{defn}

\begin{defn}
  \label{defn:bond_markets:3} $P(t, T)$is the price at time $t$ for a
  bond maturing at time $T$.
\end{defn}

\begin{defn}
  \label{defn:bond_markets:4}
  The yield $y(t, T)$ is defined as
  \begin{align}
    \label{eq:145}
    y(t, T) = - \frac{1}{T - t} \log P(t, T)
  \end{align} or equivalently
  \begin{align}
    \label{eq:146}
    P(t, T)= e^{-(T-t) y(t, T)}
  \end{align}
\end{defn}

\begin{defn}
  \label{defn:bond_markets:5}
  We call $\lim_{T \downarrow t} y(t, T) = r_{t}$ the ``spot'' or
  ``short'' rate.

  We call $\lim_{T \uparrow \infty} y(t, T)$ if it exists.
\end{defn}

\begin{defn}
  \label{defn:bond_markets:6}
  The forward rate $f(t, T)$ is defined
  \begin{align}
    \label{eq:147}
    f(t, T) = - \frac{\partial}{\partial T} \log P(t, T)
  \end{align} or equivalently
  \begin{align}
    \label{eq:148}
    P(t, T) = - \int_{t}^{T} f(t, u) du
  \end{align}
\end{defn}

\begin{thm}
  \label{defn:bond_markets:2}
  There is no arbitrage in the market prices $(P(t, T_{1}), P(t,
  T_{2}), \dots, P(t, T_{n}))$ if $Y_{t} P(t, T)_{t \in [0, T]}$ is a local
  martingale  for all $T$, where $Y$ is a state price
  density.\sidenote{Recall relative arbitrage, admissible class $D$,
    etc.}

  In particular, there is no arbitrage if $P(t, T) = \frac{1}{Y_{t}}
  \E{Y_{T} | \mathcal{F}_{t}}$
\end{thm}

Introduce the bank account $dB_{t} = B_{t} r_{t} dt \iff B_{t} = B_{0}
e^{\int_{0}^{t}r_{s} ds}$ where $r$ is the short rate.  Define an
equivalent martingale measure with density $\frac{d\Q}{d\Prob} =
\frac{B_{T} Y_{T}}{B_{0} Y_{0}}$.  Rewrite
\begin{align}
  \label{eq:149}
  P(t, T) = B_{t} \E{\frac{1}{B_{T}} | \mathcal{F}_{t}}{\Q} =
  \E{e^{-\int_{t}^{T} r_{s} ds} | \mathcal{F}_{t}}{\Q}
\end{align}

By the law of one price,
\begin{align}
  \label{eq:150}
  f(t, T) &= - \frac{\partial}{\partial T} \log \E{ e^{-\int_{t}^{T}
      r_{s} ds} | \mathcal{F}_{t}}{\Q} \\
  &= \frac{\E{r_{T} e^{-\int_{t}^{T} r_{s} ds} |
      \mathcal{F}_{t}}{\Q}}{\E{e^{-\int_{t}^{T} r_{s} ds} |
      \Phi_{t}}{\Q}},
\end{align} and so $f(t, T)$ can be seen as the ``market weighted
conditional expectation of $r_{T}$ given at $\mathcal{F}_{t}$.''

Alternatively, we have
\begin{align}
  \label{eq:151}
  \E{(f(t, T) - r_{T}) e^{-\int_{t}^{T} r_{s} ds} |
    \mathcal{F}_{t}}{\Q} = 0
\end{align} and so the forward rate is such that the claim with payout
$f(t, T) - r_{T}$ has price 0 at time $T$.


%%% Local Variables: 
%%% mode: latex
%%% TeX-master: "master"
%%% End: 
