\chapter{Discrete Time Models}
\label{cha:discrete-time-models}

\section{Standing Assumptions}
\label{sec:standing-assumptions}

\begin{enumerate}
\item Zero dividends
\item Zero tick size
\item Zero transaction costs
\item Infinitely divisible transactions
\item No short-selling constraints
\item No bid-ask spread
\item No market impact (infinitely deep market)
\end{enumerate}

\section{Setup}
\label{sec:setup}

Consider a probability space $(\Omega, \mathcal{F}, \mathbb{P})$.

\begin{defn}
  \label{defn:1}
  A random variable is a measurable map $X: \Omega \rightarrow \mathbb{R}$
\end{defn}

\begin{defn}
  \label{defn:2}
  A stochastic process $Y = (Y_{t})_{t \in I}$ is a collection of
  random variables.  For us, $I = \{ 0, 1, \dots \}$ or $[0, \infty)$. 
\end{defn}


\begin{defn}
  \label{defn:3}
  A filtration $\mathbb{F} = (\mathcal{F})_{t \geq 0}$ is a collection
  of sub-$\sigma$-algebras on $\mathcal{F}$ such that $\mathcal{F}_{s}
  \subseteq \mathcal{F}_{t}$ for all $0 \leq s \leq t$ (discrete and
  continuous time).
\end{defn}

\begin{exmp}
  Tossing coins.
  \begin{enumerate}
  \item $\Omega = \{ HH, HT, TH, TT \}$
  \item $\mathcal{F}$ is all 16 subsets of $\Omega$
  \item $\mathbb{P}(A) = \frac{|A|}{4}$
  \end{enumerate}

  Possible filtration
  \begin{enumerate}
  \item $\mathcal{F}_{0} = \{ \emptyset, \Omega \}$
  \item $\mathcal{F}_{1} = \{ \emptyset, \Omega, \{ HH, HT \}, \{ TH,
    TT \} \}$
  \item $\mathcal{F}_{2} = \mathcal{F}$
  \end{enumerate}
\end{exmp}

\begin{defn}
  \label{defn:4}
  A process $Y$ is adapted if and only if $Y_{t}$ is $\mathcal{F}_{t}$-measurable.
\end{defn}

Throughout the course, $\mathcal{F}_{0}$ is assumed trivial.

%%% Local Variables: 
%%% mode: latex
%%% TeX-master: t
%%% End: 